\documentclass{article}

% Pakete
\usepackage[utf8]{inputenc}
\usepackage[ngerman]{babel}
\usepackage{amsmath}
\usepackage{graphicx}
\usepackage{float}
\usepackage{cite}
\usepackage{multirow}


% Titel und Autor
\title{RNA-Quantifizierung mittels qPCR}
\author{Thorsten Enge \and David Maruhn}
\date{\today}

\begin{document}

% Titel erstellen
\maketitle

% Inhaltsverzeichnis
\tableofcontents
\newpage

% Einleitung
\section{Einleitung}

Während bei der DNA-Analytik untersucht wird, ob eine bestimmte
DNA-Sequenz auf dem Genom vorhanden ist, wird bei der RNA-Analytik
untersucht, ob und wie stark ein bestimmtes Gen transkribiert wird.
Das Vorhandensein eines Gens auf dem Genom bedeutet nicht zwangsläufig,
dass dieses Gen auch transkribiert wird. Mechanismen wie die
DNA-Methylierung, die Histonmodifikation, Transkriptionsfaktoren und
Enhancer bzw. Silencer beeinflussen die Transkription eines Gens.
Die durch die Analyse der RNA kann einerseits das Vorhandensein eines
Gens auf dem Genom bestätigt werden, andererseits kann durch die
quantifizierung Rückschlüsse auf die Transkriptionsrate gezogen werden
und somit auf die Aktivität des Gens. Das nichtvorhandensein einer RNA
bedeutet nicht zwangsläufig, dass das Gen nicht auf dem Genom vorhanden
ist, sondern dass es nicht transkribiert wird.


% Materialien und Methoden
\section{Materialien und Methoden}

\subsection*{Zellkulturen}
Für die RNA-Isolation wurden die Zelllinien HEK293T und A375 verwendet.
Beide Zellkulturen sind immortale humane Zelllinien.
\begin{itemize}
    \item HEK293T sind Zellen, die aus der Niere eines Embryos gewonnen wurden.
    Sie sind durch die Transformation mit dem SV40-T-Antigen immortalisiert
    worden.
    \item A375-Zellen stammen aus einem Melanom, sind sie von natur aus
    immortalisiert.
\end{itemize}

\subsection*{Genes of Interest}
Die Gene of Interest sind die Gene, die in dieser Arbeit quantifiziert
werden sollen. In diesem Fall sind es die Gene GAPDH und Beta-Actin.
\begin{itemize}
    \item GAPDH (Glyceraldehyde-3-phosphate dehydrogenase) katalysiert die 
    Oxidation von Glyceraldehyd-3-phosphat zu
    1,3-Bisphosphoglycerat im Glykolyseweg.
\[
\text{C}_3\text{H}_5\text{O}_6\text{P} + \text{NAD}^+ + \text{P}_i \xrightarrow{\text{GAPDH}} \text{C}_3\text{H}_4\text{O}_{10}\text{P}_2 + \text{NADH} + \text{H}^+
\]

    \item Beta-Actin wird in der qPCR als Referenzgen verwendet, da es in allen
    Zellen in gleicher Menge vorhanden ist und somit als Kontrollgen dient.
    
\end{itemize}


\section{Durchführung}

\subsection{RNA-Isolation mittels Monarch\textsuperscript{\textregistered} Total RNA Miniprep Kit (NEB)}

\begin{enumerate}
    \item Lysat erstellen
    \item gDNA entfernen
    \item RNA isolieren und Nebenprodukte entfernen, unter Nutzung von Ethanolpräzipitation, um die Haftung der RNA zu erleichtern.
    \item DNase 1 zerstört Reste der DNA
    \item Elution RNA in nukleasefreiem Wasser
\end{enumerate}

\subsubsection*{RNA-Isolation}
In diesem Versuch wurde die RNA aus kultivierten
humanen Zellen isoliert. Zu beachten sind die
sorgfältige Reinigung von allen Oberflächen und Geräten
mit 70\,\%{}igem Alkohol, um Verunreinigungen mit
Nukleasen zu vermeiden. Aus gleichem Grund ist es sinnvoll,
stets mit Handschuhen zu arbeiten.
Dafür wurden zunächst die tiefgefrorenen Zellpellets der
kultivierten Zellen aufgetaut, bei 4 °C für 2 Minuten bei
10.000 rpm zentrifugiert und durch Zugabe von 800 µl des
Lysepuffers in Lysat überführt. 
Durch die Lyse werden die Zellen aufgebrochen
und der Zellinhalt für die weitere Prozessierung zugänglich gemacht.
Im nächsten Schritt wurde aus dem Lysat die gDNA, die genomic DNA,
durch Waschung über einer Removal Column entfernt.
Hierfür wurden 800 µl Lysat in das gDNA Removal Column
überführt und für 1 Minute bei 10.000 rpm und
Raumtemperatur (RT) zentrifugiert.
Die Column macht sich die strukturellen Unterschiede der
DNA und RNA zunutze, um die DNA zu fixieren. Der flüssige
Durchsatz wurde nach Entfernen und Entsorgung der gDNA Removal
Column im Verhältnis 1:1 mit Ethanol (\(\geq\) 95\%) gemischt.
Die Mischung wurde in mehreren Durchgängen in die RNA Purification
Column überführt und für 1 Minute, 10.000 rpm und RT zentrifugiert.
Der Durchsatz wurde nach jedem Durchgang verworfen. Der Prozess
wurde wiederholt, bis die Mischung einmal komplett durch die RNA Removal
Column zentrifugiert wurde. 
Anschließend wurde die RNA Removal Column mit 500 µl RNA Wash Buffer
benetzt und nochmals für 1 Minute zentrifugiert. Der Durchsatz wurde verworfen.
Im nächsten Schritt wurde die RNA Removal Column in ein Tube
überführt, das 75 µl DNAse I Reaction Buffer enthält und die Matrix
mit DNAse I benetzt. 
Das Tube wurde 15 Minuten bei RT inkubiert. Danach wurden 500 µl Wash
Buffer auf die Säule pipettiert, 1 Minute zentrifugiert und der
Überstand verworfen. Anschließend wurden 500 µl Priming Buffer auf die
Säule pipettiert, 1 Minute zentrifugiert und der Überstand verworfen.
Es folgte eine Waschung mit 500 µl Wash Buffer und 1 Minute Zentrifugation.
Nach Verwerfen des Überstands wurde nochmals mit 500 µl Wash Buffer auf die
Säule pipettiert, 2 Minuten zentrifugiert und der Überstand entfernt.
Die Säule wurde in ein nukleasefreies Tube überführt.
Es folgte die Elution der RNA, indem 50 µl nukleasefreies
H$_2$O auf die Matrix der Säule pipettiert und 1 Minute zentrifugiert wurde.
Die genaue Vorgehensweise ist der Anleitung des Monarch
Total RNA Miniprep Kit (NEB) zu entnehmen.

\subsubsection*{Spektroskopie zur Bestimmung der Nukleinsäurenkonzentration}

Dieser Abschnitt findet statt, um die Qualität der
Aufbereitung der RNA aus den Zellpellets zu bewerten.
Hierfür wurde 1µl der Probe vermessen. Ziel war es,
das Verhältnis der Absorption der Wellenlänge 260/280 [nm] zu überprüfen.
Der experimentelle Schwellwert für eine weiterverwendbare RNA-Konzentration
in der Probe ist ca. 2,0. Die Messung fand in einem Spektrophotometer
\mbox{DS-11+ -DeNovix} statt. Benutzt wurde das RNA-Programm.
Zunächst musste eine blank-messung durchgeführt werden,
um dem Spektrophotometer einen Bezugsrahmen für folgende Messungen zu geben.
Dies erfolgt durch die Messung 1 µl nukleasefreies H$_2$O. Die Messung
erfolgte über die Mikrovolumenabsorptionseinheit. Nach der Eichung des
Referenzrahmens folgte die Messung der einzelnen Proben. Stets mit 1 µl
Probe auf der Mikrovolumenabsorptionseinheit, die nach jeder Messung mit
der gleichen Menge nukleasefreiem H$_2$O gereinigt wurde. 

\subsection{cDNA-Synthese aus isolierter RNA}

\begin{itemize}
    \item Mischen der Komponenten in spezifischer Reihenfolge
    \item Inkubation
\end{itemize}

\subsubsection*{Reverse Transkription - cDNA-Synthese aus isolierter RNA}

Zunächst wurde ein Heizblock TS pro - CellMedia auf 25 °C,
5 Minuten und nicht schüttelnd voreingestellt. Der Puffer,
dNTPs, Random Primer Mix und RNA wurden auf Eis aufgetaut.
RNAse Inhibitor und M-MuLV Reverse Transkriptase wurden bei -20 °C gelagert.
Für die RT-PCR wurde ein Reaktionsvolumen von 20 µl festgelegt.

\begin{table}[H]
\centering
\begin{tabular}{|l|l|}
\hline
\textbf{Komponenten} & \textbf{Volumen} \\ \hline
Random Primer Mix & 2 µl \\ \hline
10X M-MuLV RT Puffer & 2 µl \\ \hline
10 mM dNTP & 1 µl \\ \hline
RNAse Inhibitor (40 U/µl) & 0.2 µl \\ \hline
M-MuLV Reverse Transkriptase (200 U/µl) & 1 µl \\ \hline
RNA (5 µg) & Probenabhängige Konzentration \\ \hline
Nukleasefreies H\textsubscript{2}O & Bis 20 µl auffüllen \\ \hline
\end{tabular}
\caption{Komponenten für die cDNA-Synthese}
\end{table}

Die angegebenen Komponenten wurden in ein 1,5 ml
Eppendorf-Gefäß gefüllt und abzentrifugiert.
Abschließend wurde die Inkubation im Heizblock vorgenommen.
Diese erfolgte in drei Schritten:

\begin{table}[H]
    \centering
    \begin{tabular}{|c|c|}
    \hline
    \textbf{Zeit} & \textbf{Temperatur} \\ \hline
    5 Minuten & 25 °C \\ \hline
    60 Minuten & 42 °C \\ \hline
    20 Minuten & 65 °C \\ \hline
    \end{tabular}
    \caption{Inkubationszeiten und Temperaturen}
    \end{table}

\subsection{Quantifizierung durch qPCR (Real-Time PCR)}

\begin{itemize}
    \item 3 Mastermixe erstellen (je Primer-Paar)
    \item PCR im Cycler
\end{itemize}

\subsubsection*{Quantifizierung durch qPCR}
Zur Untersuchung von drei Primern wurde jeweils ein
Master Mix mit 80 µl Reaktionsvolumen für eine Messung,
eine Zweitmessung und eine NTC in 20 µl vorbereitet.
Jeder Master Mix enthielt pro Reaktion (20 µl) 10 µl 2X Biozym HRM Mix,
0.4 µl 10 µM Primer forward, 0.4 µl 10 µM Primer reverse, 8.2 µl
nukleasefreies H\textsubscript{2}O und 1 µl cDNA oder
nukleasefreies H\textsubscript{2}O für die NTC. Die Master Mixe
wurden in PCR-Tubes transferiert. Die cDNA und das
alternative H\textsubscript{2}O wurden als letztes pipettiert.
Anschließend wurden die PCR-Tubes verschlossen und auf Eis gelegt.
Die Proben wurden in einem Roche Light Cycler 96 mit folgendem
Programm analysiert:

\begin{table}[H]
    \centering
    \begin{tabular}{|l|l|l|l|}
    \hline
    \textbf{Cycles} & \textbf{Step} & \textbf{Temp [°C]} & \textbf{Time [sec]} \\ \hline
    1 & Preincubation & 95 & 120 \\ \hline
    \multirow{2}{*}{40} & \multirow{2}{*}{2 Step Amplification} & 95 & 5 \\ \cline{3-4}
     &  & 65 & 30 \\ \hline
    \multirow{4}{*}{1} & \multirow{4}{*}{High Resolution Melting} & 95 & 60 \\ \cline{3-4}
     &  & 40 & 60 \\ \cline{3-4}
     &  & 65 & 1 \\ \cline{3-4}
     &  & 97 & 1 \\ \hline
    1 & Cooling & 37 & 30 \\ \hline
    \end{tabular}
    \caption{PCR-Zyklusprogramm}
    \end{table}
  


% Diskussion
\section{Diskussion}

% Schlussfolgerung
\section{Schlussfolgerung}

% Literaturverzeichnis
\bibliographystyle{plain}
\bibliography{literatur}

\end{document}
