\documentclass{article}

% Pakete
\usepackage[utf8]{inputenc}
\usepackage[ngerman]{babel}
\usepackage{amsmath}
\usepackage{graphicx}
\usepackage{float}
\usepackage{cite}

% Titel und Autor
\title{RNA-Quantifizierung mittels qPCR}
\author{Thorsten Enge \and David Maruhn}
\date{\today}

\begin{document}

% Titel erstellen
\maketitle

% Inhaltsverzeichnis
\tableofcontents
\newpage

% Einleitung
\section{Einleitung}

Während bei der DNA-Analytik untersucht wird, ob eine bestimmte
DNA-Sequenz auf dem Genom vorhanden ist, wird bei der RNA-Analytik
untersucht, ob und wie stark ein bestimmtes Gen transkribiert wird.
Das Vorhandensein eines Gens auf dem Genom bedeutet nicht zwangsläufig,
dass dieses Gen auch transkribiert wird. Mechanismen wie die
DNA-Methylierung, die Histonmodifikation, Transkriptionsfaktoren und
Enhancer bzw. Silencer beeinflussen die Transkription eines Gens.
Die durch die Analyse der RNA kann einerseits das Vorhandensein eines
Gens auf dem Genom bestätigt werden, andererseits kann durch die
quantifizierung Rückschlüsse auf die Transkriptionsrate gezogen werden
und somit auf die Aktivität des Gens. Das nichtvorhandensein einer RNA
bedeutet nicht zwangsläufig, dass das Gen nicht auf dem Genom vorhanden
ist, sondern dass es nicht transkribiert wird.


% Materialien und Methoden
\section{Materialien und Methoden}

\subsection*{Zellkulturen}
Für die RNA-Isolation wurden die Zelllinien HEK293T und A375 verwendet.
Beide Zellkulturen sind immortale humane Zelllinien.
\begin{itemize}
    \item HEK293T sind Zellen, die aus der Niere eines Embryos gewonnen wurden.
    Sie sind durch die Transformation mit dem SV40-T-Antigen immortalisiert
    worden.
    \item A375-Zellen stammen aus einem Melanom, sind sie von natur aus
    immortalisiert.
\end{itemize}

\subsection*{Genes of Interest}
Die Gene of Interest sind die Gene, die in dieser Arbeit quantifiziert
werden sollen. In diesem Fall sind es die Gene GAPDH und Beta-Actin.
\begin{itemize}
    \item GAPDH (Glyceraldehyde-3-phosphate dehydrogenase) katalysiert die 
    Oxidation von Glyceraldehyd-3-phosphat zu
    1,3-Bisphosphoglycerat im Glykolyseweg.
\[
\text{C}_3\text{H}_5\text{O}_6\text{P} + \text{NAD}^+ + \text{P}_i \xrightarrow{\text{GAPDH}} \text{C}_3\text{H}_4\text{O}_{10}\text{P}_2 + \text{NADH} + \text{H}^+
\]

    \item Beta-Actin wird in der qPCR als Referenzgen verwendet, da es in allen
    Zellen in gleicher Menge vorhanden ist und somit als Kontrollgen dient.
    
\end{itemize}



\section{Durchführung}

% Ergebnisse
\section{Ergebnisse}

\begin{table}[H]
    \centering
    \begin{tabular}{|c|c|c|}
    \hline
    Messung & Wert1 & Wert2 \\
    \hline
    1 & 10 & 20 \\
    2 & 30 & 40 \\
    \hline
    \end{tabular}
    \caption{Beispieltabelle}
    \label{tab:beispiel}
\end{table}


% Diskussion
\section{Diskussion}

% Schlussfolgerung
\section{Schlussfolgerung}

% Literaturverzeichnis
\bibliographystyle{plain}
\bibliography{literatur}

\end{document}
